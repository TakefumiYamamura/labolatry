%% 和文論文用のテンプレート
%%%%%%%%%%%%%%%%%%%%%%%%%%%%%%%%%%%%%%%%%%%%%%%%%%%%%%%%%%%%%%%%%%%%%%%%%%%%%%%
% 以下のうち,対応する一つのみコメント ('%') を外してください.
\documentclass[originalpaper]{jpnsecart}      % 論文 Original Paper
% \documentclass[invitedpaper]{jpnsecart}      % 招待論文 Invited Paper
% \documentclass[surveypaper]{jpnsecart}       % 解説 Survey Paper
%\documentclass[practicalpaper]{jpnsecart}       % 事例紹介論文 Practical Application Paper

%% amsmathパッケージの注意点 %%%%%%%%%%%%%%%%%%%%%%%%%%%%%%%%%%%%%%%%%%%%%%%%%%
% \usepackpage{amsmath}
% 数式番号の参照は \ref ではなく,\eqref を使ってください.
% documentclass のオプションに fleqnを指定してください.
% 例: \documentclass[technicalpaper,fleqn]{jpnsecart}

%% ページ番号,Volume,Number %%%%%%%%%%%%%%%%%%%%%%%%%%%%%%%%%%%%%%%%%%%%%%%%%
% 掲載時に学会事務局が指定します.
\received{2010}{8}{1}
\stuffincharge{XX\hspace{0.5zh} XX}
\setcounter{page}{1}
%\setcounter{volpage}{1}
\Vol{1}
\No{1}

%% タイトル %%%%%%%%%%%%%%%%%%%%%%%%%%%%%%%%%%%%%%%%%%%%%%%%%%%%%%%%%%%%%%%%%%%
\jtitle{適応DEにおけるアーカイブ性能の解析}
% \jtitle[柱用和文タイトル]{和文タイトル}
\jsubtitle{和文サブタイトル}
\etitle{The Improvement Of the Archive in Adaptated Differential Evolution}
\esubtitle{英文サブタイトル}

%% 著者名 %%%%%%%%%%%%%%%%%%%%%%%%%%%%%%%%%%%%%%%%%%%%%%%%%%%%%%%%%%%%%%%%%%%%%
% 所属先が同じ著者が連続する場合,その中の先頭の著者のみ \affiliation
% を使って,残りの所属先には \sameaffiliation を使ってください.
% ただし,所属先が同じでも連続していない場合は \affliation を使ってください.
% 名前が長い場合は \name の代りに \longname を使ってください.
% 著者が3名以下の場合は以下の \manyauthor をコメントアウトしてください.
\manyauthor
\author{%
 \name{山村}{武史}{Yamamura Takefumi}
 \affiliation{東京大学 教養学部 学際科学科総合情報学コース}%
     {The Department of Interdisciply Sciences, The University of Tokyo}%
     {yama1223xxx@gmail.com, http://www.ids.c.u-tokyo.ac.jp/info/}
\and
 \name{福永}{アレックス}{Fukunaga Alex}
 \affiliation{東京大学大学院総合文化研究科 広域科学専攻 広域システム科学系 }%
     { Department of General Systems Studies, Graduate School of Arts and Sciences, The University of Tokyo}%
     {fukunaga@idea.c.u-tokyo.ac.jp, http://metahack.org/index-j.html}
}

%% Keywords, Summary %%%%%%%%%%%%%%%%%%%%%%%%%%%%%%%%%%%%%%%%%%%%%%%%%%%%%%%%%%
\begin{keyword}
Artificial Inteligence, Evolutionary Algorithm, Genetic Algorithm, Differential Evolution, Archive
\end{keyword}

\begin{summary}
Differential Evolution is a simple, but effective approach for numerical optimization. This approach can find a local minimum and can solve the any numerical optimization in independent problem.
200〜500 ワード以内の英文でsummaryを書いてください.
\end{summary}

%% 本文 %%%%%%%%%%%%%%%%%%%%%%%%%%%%%%%%%%%%%%%%%%%%%%%%%%%%%%%%%%%%%%%%%%%%%%%
\begin{document}
\maketitle

\section{はじめに}
実数値最適化問題とはD次元の実数値ベクトル$\vec{x} = (x_1, x_2, \cdots, x_D)$ が存在する時に有る評価関数${f(x)}$を最小にするような最適解$x$を求める問題である.実数値最適化問題は多峰性,変数間依存性や悪スケール性を考慮して解候補	
Differential Evolution(DE) は主に連続最適化問題を対象としたEvolutionary Algorithm(EA)である.
DEは単純なアルゴリズムで有るにも関わらず,優れた探索性能をもつことから様々な実問題に適用されている.

DEでは大域的探索を可能にするためには,解集合における多様性が保証するためアーカイブが用いられることが多い.アーカイブは進化計算における生存選択の際に劣った親個体を,アーカイブに保持し,差分ベクトル生成時にその劣った親個体を利用することで解探索における多様性を維持する.state of the artなDEの多くでアーカイブを使用することが多いがその実性能について厳密に議論されたことは未だ少ない.本論文ではDEにおけるアーカイブが現実でどの程度有用であるかを複数の手法を用いて検証する.
検証すべき仮定
アーカイブが多様性をどれだけ維持できているのか,分散をアーカイブがありとアーカイブなしで各世代ごとに比べる
アーカイブによる解更新が起きた際は普通の突然変異のステップ幅よりも大きく,通常の突然変異より大きな変化を生じる.またその際にわかる解は大域的な解であり,
巡
多様性を維持するため,突然変異において解集合と離れた位置に差分ベクトル分移動する
アーカイブに劣っていた親個体を保持し,変異ベクトル作成時に使用することでその多様性を維持する.
DEにおける可能な
仮説2
解探索における集団的降下特性が,アーカイブ性能を下げる方向に働いているのでは?
集団的降下特性とは

仮説3
DEの性能はスケーリングパラメーターF,交叉率CR,集団数Nに大きく依存し,適切なパラメータは解くべき問題によって大きく異なる.
アーカイブの範囲,サイズといったパラメータ値も解くべき問題によって異なるのではないか?

仮説4
アーカイブはDEの性能を大きく左右するF,CR,集団数Nに比べてその影響が小さい.
その原因として考えられる要因
適切なアーカイブが選択されていない.
そもそ適切なアーカイブとは?
ステップ幅を最適な値に調整する,

仮説5
同一の問題においても探索状況に応じて最も適応するアーカイブのパラメータは異なる値を示す.

仮説6
アーカイブでは解探索において,通常の突然変異より大きな幅の変異を生じる.したがって,各世代ごとに分散を見てあげれば,アーカイブによって多様性が維持されていることが確認出来る.


\section{従来の手法の問題点}
\section{今回の改良点}
\section{実験}
\subsection{実験手法}
\subsection{実験結果}
\subsubsection{実験1}
\subsubsection{実験2}
\section{まとめ}

\begin{acknowledgment}
謝辞について
\end{acknowledgment}

\bibliography{btxsample}
\bibliographystyle{jpnsec}

\appendix

\section{付録のタイトル1}

付録の本文1

\section{付録のタイトル2}

付録の本文2

%% 著者紹介 %%%%%%%%%%%%%%%%%%%%%%%%%%%%%%%%%%%%%%%%%%%%%%%%%%%%%%%%%%%%%%%%%%%
% 著者の姓と名の間は半角スペースで区切ってください.
% 略歴は200字以内で書いてください.
\begin{biography}
\profile{m}{著者1姓 名}{著者1の略歴}
\end{biography}

\end{document}
