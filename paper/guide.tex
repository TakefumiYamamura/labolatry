\documentclass{jpnsec2e}

\usepackage{graphics}
\usepackage{fancybox}
\usepackage{comment}

%%%%% LaTeXなどのロゴの定義
\makeatletter
\catcode`\@=11
\font\eightsy=cmsy8
\def\AMSTeX{\leavevmode
\hbox{$\scriptstyle\cal A$}\kern-.25em
 \lower.4ex\hbox{\eightsy M}\kern-.1em{$\cal S$}-\TeX}
\catcode`\@=\active
\def\LaTeX{{\rm L\kern-.36em\raise.3ex\hbox{$\scriptstyle {\rm A}$}\kern-.15em
 T\kern-.1667em\lower.7ex\hbox{E}\kern-.125emX}}
\def\JTeX{\leavevmode\lower.5ex\hbox{J}\kern-.17em\TeX}
\def\BibTeX{{\rm B\kern-.05em{\sc i\kern-.025em b}\kern-.08em%
 T\kern-.1667em\lower.7ex\hbox{E}\kern-.125emX}}
\def\StyleFile{{\tt jpnsec2e.cls}}\tt
\def\JBibTeX{\leavevmode\lower .6ex\hbox{J}\kern-0.15em\BibTeX}
\def\LaTeXe{\LaTeX\kern.15em2$_{\textstyle\varepsilon}$}
\def\Stylefile2e{{\tt jpnsec2e.cls}}
\makeatother

%%%%% その他のマクロ
\verbatimbaselineskip.5\baselineskip
\def\verbatimsize{\small}%%{\normalsize}
\def\tbs{{\tt \char'134}} % \
\def\tob{{\tt \char'173}} % {
\def\tcb{{\tt \char'175}} % }
\newcommand{\bsl}[1]{{$\mathtt{\backslash}$}\texttt{#1}}
\newcommand{\ltt}[1]{\texttt{\small{}#1}}

\newenvironment{verbfbox}%
{\VerbatimEnvironment%
\begin{Sbox}\begin{minipage}{79mm}\footnotesize\begin{Verbatim}}%
{\end{Verbatim}\end{minipage}\end{Sbox}
\cornersize*{3mm}\setlength{\fboxsep}{1mm}\par\noindent\ovalbox{\TheSbox}}

\newenvironment{verbFbox}%
{\VerbatimEnvironment%
\begin{Sbox}\begin{minipage}{77mm}\small\begin{Verbatim}}%
{\end{Verbatim}\end{minipage}\end{Sbox}%
\setlength{\fboxsep}{1.4mm}\par\noindent\doublebox{\TheSbox}\par}

\raggedbottom
\pagestyle{plain}

\jtitle{進化計算学会\LaTeX{}スタイルファイルの使い方}
\etitle{How to Use \LaTeX{} Style Files}

\author{%
\name{}{}{}
\affiliation{進化計算学会編集委員会}%
{Editorial Board, Japanese Society for Evolutionary Computation}%
{editor@jpnsec.org,
http://www.jpnsec.org/journal.html}
}

\rm
\begin{summary}
This is a guide to the style files 
for the Journal of Japanese Society for Evolutionary Computation. 
\end{summary}

\begin{document}
\maketitle

\section{はじめに}\label{sec:intro}

このスタイルファイルは,進化計算学会論文誌の原
稿を作成するためのものです.
アスキー版日本語 p\TeX{}のVersion p2.1.5以降を対象としています.


このスタイルファイルは,本誌の組版体裁にしたがって調整されているので,
スタイルファイルの変更は一切しないでください.

本誌で使われる\LaTeXe 用のクラスファイルとテンプレートは,次のとおりです.
\medskip

{\small
\begin{tabular}{ll}
\hline
\ltt{jpnsecart.cls}    & 論文用クラスファイル\\
\ltt{jpnsec2e.cls}     & 補助クラスファイル\\
\ltt{jpnsec.bst}      & 参考文献用クラスファイル\\
\ltt{profile-2e.sty}   & 著者紹介用クラスファイル\\\hline
\ltt{template-j.tex}  & 日本語論文用テンプレート\\
\ltt{template-e.tex}   & 英語論文用テンプレート\\\hline
\end{tabular}
}
\medskip

論文用クラスファイル\ltt{jpnsecart.cls} と補助クラスファイル\ltt{jpnsec2e.cls}は同じ場所においてください.


\ref{japanese}と\ref{english}では,それぞれ
日本語の原稿,英語の原稿についての書式と注意事項を述べます.
\ref{sec:generalnote} では,句読点,脚注,相互参照,拡張マクロについ
ての注意事項を述べます.
\ref{sec:figtab}では,図表の注意事項,\textsc{PostScript}ファイルの取り
込みに関する規定などを述べます.
\ref{sec:equation}では,数式についての注意事項を述べます.\ltt{amsmath}を用いる場合は,特に注意が必要です.
\ref{sec:bib}では,参考文献についての注意事項を述べます.
\ref{submit}では,採録決定後に提出するファイルについて述べます.

\section{日本語の原稿について}\label{japanese}

\ltt{jpnsecart.cls} を用いて,オプション(\verb|[]|内)に次の
ことを指定してください.
\medskip

\begin{center}
\small
\begin{tabular}{@{}ll@{}}
\hline
\noalign{\vskip1mm}
\tt{originalpaper}   & 論文(Original Paper)\\
\tt{invitedpaper}    & 招待論文(Invited Paper)\\
\tt{surveypaper}     & 解説(Survey Paper)\\

\noalign{\vskip1mm}
\hline
\end{tabular}
\end{center}
\medskip
\noindent
例えば,論文であれば, 次のように指定してください.
\begin{verbfbox}
\documentclass[originalpaper]{jpnsecart}
\end{verbfbox}
\noindent

\subsection{原稿の書き方}

原稿の全体構成を\ref{fig:原稿構成}に示します.
テンプレートファイル \ltt{template-j.tex} をベースに,これを編集して原稿を作成してください.

\begin{figure}[p]
\begin{verbFbox}
\documentclass[originalpaper]{jpnsecart}

\Vol{1}   %%掲載論文の巻番号;事務局が指定
\No{1}    %%掲載論文の号番号;事務局が指定

\jtitle{日本語タイトル}

\etitle{英語タイトル}

\author{%
 \name{姓}{名}{ローマ字読み}
 \affiliation{日本語所属名}{英語所属名}%
                             {e-mail,URL}
\and
 \name{姓}{名}{ローマ字読み}
 \affiliation{日本語所属名}{英語所属名}%
                             {e-mail,URL}
}

\begin{keyword}
keywords in English
\end{keyword}

\begin{summary}
summary in English
\end{summary}
%\setcounter{page}{1}

\begin{document}
\maketitle
\section{はじめに}
%% 本文の内容%%

\begin{acknowledgment}
%% 謝辞の内容%%
\end{acknowledgment}

\begin{thebibliography}{??}
\bibitem[]{}
\bibitem[]{}
\end{thebibliography}

\appendix
\section{付録}
%% 付録の内容%%

\begin{biography}
\profile{会員種別}{著者名}{略歴内容}
\end{biography}

\end{document}
\end{verbFbox}
\caption{原稿の構成}
\label{fig:原稿構成}
\end{figure}

\subsubsection{\texttt{jtitle}}
日本語のタイトルを書いてください.
タイトル中に改行(\verb|\\|)を指定すれば,タイトル中で改行できま
すが,柱\footnote{ヘッダ.ページ上部のページ数などがある部分}
では無視されます.長すぎて,柱に収まらない場合,
\begin{verbfbox}
\jtitle[柱用タイトル]{タイトル}
\end{verbfbox}
\noindent
のようにすれば,柱には\verb|[]|内のものが使われます.

\subsubsection{\texttt{etitle}}
英語のタイトルを書いてください.
前置詞,接続詞,文中冠詞などを除いて,単語の先頭文字は大文字にしてください.

\subsubsection{\texttt{jsubtitle}と\texttt{esubtitle}}
サブタイトルがある場合は,これらを用いて指定してください.
\verb/\jsubtitle/ は日本語用,\verb/\esubtitle/ は英語用です.
これらは柱には出力されません.

\subsubsection{\texttt{author},\texttt{name},\texttt{affiliation}}

著者の姓名,所属名などを,以下のように指定してください.名前が長い方は
\verb/\name/の代わりに
\verb/\longname/を使ってください.
\begin{verbfbox}
\author{%
 \name{姓}{名}{ローマ字読み}
 \affiliation{日本語所属名}{英語所属名}{e-mail,URL}
}
\end{verbfbox}

\verb/\affiliation/ の第3引数には著者のe-mailアドレスを書き,
ホームページがある場合には,``,''のあとにURLを書いてください.
URL中の``\~{}'' はそのまま(エスケープしないで)記してください.

著者が複数の場合,\verb|\and| を用いて,次のように書いてください.
\begin{verbfbox}
\author{%
 \name{姓1}{名1}{ローマ字読み1}
 \affiliation{日本語所属1}{英語所属1}{e-mail,URL}
\and
 \name{姓2}{名2}{ローマ字読み2}
 \sameaffiliation{e-mail,URL}
\and
 \longname{姓3}{名3}{ローマ字読み3}
 \affiliation{日本語所属3}{英語所属3}{e-mail,URL}
\and
 \longname{姓4}{名4}{ローマ字読み4}
 \affiliation{日本語所属1}{英語所属1}{e-mail,URL}
}
\end{verbfbox}
\noindent

同じ所属の著者が連続する場合,\verb/\affiliation/の代わりに
\verb/\sameaffiliation/を用いて,メールアドレスとURLだけを
指定してください.上の例では,著者1と2は所属が同じため著者2
は\verb/\sameaffiliation/を利用しています.


著者4名以上の場合,\verb/\author/の前に
\verb/\manyauthor/をおいて,著者名の行間を詰めてください.

\subsubsection{\texttt{keyword}}

3〜5 語の英単語を,略語や固有名詞などの場合を除いて,小文字で
列挙してください.

\subsubsection{\texttt{summary}}
要約を200〜500 \texttt{words}の英文で書いてください.

\subsubsection{document}

\texttt{summary}までを記述したあとに,
\begin{verbfbox}
\begin{document}
\maketitle
\end{verbfbox}
\noindent
と指定してから,本文以下を書いてください.

\subsubsection{\texttt{acknowledgment}}

謝辞は\verb/\acknowledgment/ を用いて書いてください.

\subsubsection{\texttt{thebiography}}

参考文献は\verb/\bibtem/ を用いて書いてください.
参考文献の書き方は\ref{sec:bib}を参照してください.

\subsubsection{appendix}

付録は\verb/\appendix/ を用いて書いてください.付録における
\verb|\section|の番号は(A.1),(A.2) $\cdots$ となります.

\subsubsection{biography}\label{profile}

著者紹介は,次のように書いてください.
\begin{verbfbox}
\begin{biography}
\profile{m}{進化 太郎}
{19xx年xx月XX大学XX学部XX学科卒業.原稿の内容(省略)}
\end{biography}
\end{verbfbox}
\noindent
第1引数には一般会員,学生会員などの会員種別を,次表にしたがって,\ltt{m},\ltt{s},\ltt{n}のいずれかで指定してください.
\medskip

\begin{center}
\begin{small}
\begin{tabular}{cll}
\hline
\noalign{\vskip1mm}
指定する文字 & 日本語の場合 & \multicolumn{1}{c}{英語の場合} \\
\noalign{\vskip1mm}
\hline
\noalign{\vskip1mm}
\ltt{m} & 正会員   & Member \\
\ltt{s} & 学生会員 & Student Member \\
\ltt{n} & (なし) & (なし) \\
\noalign{\vskip1mm}
\hline
\noalign{\vskip1mm}
%%\multicolumn{3}{l}{\parbox[t]{20zw}{}}
\end{tabular}
\end{small}
\end{center}
\medskip

第2引数の著者名は,姓と名の間を半角のスペース \verb*| |で
区切って記してください.第3引数の略歴は,200字以内で書いてください.

\subsubsection{論文受理日,担当委員}
\verb/\received/ は, \verb/\received{2010}{4}{1}/ のように論文受理日を指定します.
\verb/\stuffincharge/ は, \verb/\stuffincharge{進化 \hspace{0.5zh} 花子}/ \\
のように担当委員を指定します.

これらは,論文の採録決定後に事務局が指定します.

\subsubsection{巻番号,号番号,ページ番号}
\ltt{Vol} は \verb/\Vol{1}/ のように巻番号を指定します.
\verb/\No/ は \verb/\No{1}/ のように号番号を指定します.
\verb/\page/ は \verb/\page{1}/ のように論文のページ番号を指定します.

これらは,論文の採録決定後に事務局が指定します.


\section{英語原稿について}\label{english}

英語原稿の場合は,\verb|\documentclass|の
オプション(\verb|[]|内)を用いて,\ltt{english} を指定してください.
\verb|\name|や\verb|\affiliatiton|の書式は,次のようになります.
\begin{verbfbox}
\author{%
 \name{First Name}{Last Name}
 \affiliatiton{英語所属名}{e-mail,URL}
}
\end{verbfbox}


\section{原稿全般に関する注意事項}\label{sec:generalnote}
\subsection{句読点}

日本語の句読点はカンマ(\makebox[1zw][c]{\hbox{,}})と
ピリオド(\makebox[1zw][c]{\hbox{.}})を使い,
``\makebox[1zw][c]{\hbox{、}}'' と ``\makebox[1zw][c]{\hbox{。}}''は
使わないでください.
半角の句読点は,数式や英文中でのみを使い,日本語文中では全角の句
読点を使ってください.

2倍ダッシュ(ダーシ)の``\ddash''は,英文中を除いて,日本語の中では 
\verb/\ddash/ を使ってください.

\subsection{脚注}\label{sec:footnote}


脚注マークは,カウンターが進むごとに $\ast$1,$\ast$2,$\ast$3 となります.


\verb|\section{}| や \verb|\subsection{}| などの中では脚注は利用で
きません.

タイトル中で脚注をつける場合は,次のように手動でカウンタの値を調節
する必要があります.


\begin{verbfbox}
\begin{document}
\maketitle
\footnotetext[1]{現職:進化計算研究所}
\setcounter{footnote}{1}
\end{verbfbox}
\bsl{footnotemark[}$\langle$脚注番号$\rangle$\texttt{]}
で番号を付けて,内容は
\bsl{footnotetext[}$\langle$脚注番号$\rangle$\texttt{]\{}
$\langle$脚注の内容$\rangle$\texttt{\}}
によって記述します.そのあとにカウンタ\ltt{footnote}をタイト
ル中で用いた最後の脚注番号にします.これらは\bsl{maketitle}の直後
に書いてください.

\subsection{相互参照}

図表の相互参照は,図表環境内に\verb/\ref{fig:1}/ のように指定すれば,
``図1''(英語では Figure~1),``表1''(英語では Table~1)と
出力されます.

数式は,\verb|\label{eq:01}| のようにラベルをつけておけば,
\verb|\ref{eq:01}| によって参照することができます.
参照箇所では,括弧を明示的につけなくても,(1)のように出力され
ます.
ただし,\ltt{amsmath}スタイルを利用する場合は,注意すべき点があるので,
\ref{amstex}を参照してください.

\def\DOT{\hbox to .5zw{\hss {\rm ・}\hss}}%
sectionの番号を参照すると,``1章''(英語では ``Chapter~1''),
subsectionは``1\DOT{}1節''(英語では ``Section~1\DOT{}1''),subsubsectionは
``1\DOT 1\DOT 1節''(英語では``Section~1\DOT 1\DOT 1'') 
のように,\verb|\ref|だけで``章''や``節''が補われます.

その他の参照は,番号のみが出力されるので,出力結果に合わせるように
してください.

\subsection{拡張マクロ}

次の拡張マクロがあります.
\medskip

\begin{center}
\begin{small}
\begin{tabular}{@{}ll@{}}
\hline
\noalign{\vskip1mm}
\verb/\QED/ & 「証明終」の($\Box$)\\
\verb|\MARU{1}$\sim$\MARU{5}| & \MARU{1}$\sim$\MARU{5} \\
\verb|\kintou{4zw}{時間}| & 均等割り付け:\kintou{3zw}{時間} \\
\verb|\ruby{閾}{しきい}値| & ルビ:\ruby{閾}{しきい}値 \\
\verb/\onelineskip/ & 1行アキ \\
\verb/\halflineskip/ & 半行アキ \\
\noalign{\vskip1mm}
\hline
\end{tabular}
\end{small}
\end{center}
\smallskip
通常の\LaTeX{}では,\verb|\,| や \verb|\;| は数式中でしか使えませんが,本スタイルファイルでは数式以外でも使えます.

また,\verb|\section{}| や \verb|\subsection{}| 
の中では\verb|\ % $ # _| などが使えませんが,
次のような方法で使うことができます.
\medskip

\begin{verbfbox}
\def\tbs{\ltt{\char'134}} % \
\section{コマンド \texttt{\tbs \char} について}
\end{verbfbox}

\section{図表}\label{sec:figtab}

図表の出力位置を指定するオプションは,\ltt{h} は使わないで,
\ltt{t},\ltt{b},\ltt{tbp} などを指定して,ページの上端か下端に配
置してください.
表のキャプションは表の上に,図のキャプションは図の下に書いてください.

キャプションの幅を図表の幅に合わせたい場合には,
\verb|\capwidth|を使って,次のように指定してください.
\begin{verbfbox}
\begin{figure*}[t]
\begin{center}
\epsfile{file=xxxx.eps,width=90mm}
\end{center}
\capwidth=90mm %
\caption{図の説明文 ... }
\end{figure*}
\end{verbfbox}

取り込みが可能な図の形式はepsファイルのみです.
取り込みには\ltt{graphics}パッケージまたは\ltt{eclepsf.sty},
\ltt{epsbox.sty},\ltt{epsf.sty}
スタイルファイルのいずれかを使ってください.
これらの使い方については,\cite{latex-graphics-companion}や
\cite{中野}を参照してください.

\textsc{PostScript}ファイル中では以下の
PSフォントのみを用いてください.
\begin{verbfbox}
Courier, Courier-Bold, 
    Courier-Oblique, Courier-BoldOblique,
Helvetica, Helvetica-Bold, 
    Helvetica-Oblique, Helvetica-BoldOblique, 
Times, Times-Bold, Times-Italic, Times-BoldItalic
Symbol, ZapfDingbats,
中ゴシックBBB, リュウミンライトKL
\end{verbfbox}
その他のPS,TrueTpye,OpenTypeのフォントを用いる場合は,必ずアウトライン化してください.

文字を含む線画を取り込む場合,本文の文字の大きさとのバランスが取れるように,
文字の大きさや線の太さを調整してください.

\section{数式}\label{sec:equation}

\subsection{独立行の数式}

独立行の数式の記述には, \ltt{\$\$} ではなく,\bsl{[} や \ltt{equation}環
境を使ってください.

\StyleFile{}には \ltt{fleqn.sty} が組み込まれており,数式は
左寄せで出力されます.
数式は,文書の幅をはみ出しやすいので,特に注意してください.

\subsection{アメリカ数学会のスタイルファイルの使用}
\label{amstex}

\ltt{amsfonts}スタイルなどを用いて,以下のフォントが利用できます.
\begin{verbfbox}
msam5, msam6, msam7, msam8, msam9, msam10
msbm5, msbm6, msbm7, msbm8, msbm9, msbm10
\end{verbfbox}

\ltt{amsmath}スタイルファイルを用いる場合は
\begin{itemize}
\item \bsl{documentclass}のオプションに\ltt{fleqn}を指定してください
\item 数式番号の参照は,\ltt{amsmath}の
\bsl{eqref}を用いてください
\end{itemize}

その他,\ltt{amsmath}スタイルファイルの詳細は,
\cite{latex-companion}や\cite{中野}を参照してください.

\subsection{$\backslash$\texttt{newtheorem}について}

\verb/\newtheorem/ は,本誌の体裁にしたがって調整してあります.
日本語モード,英語モードそれぞれに,\ref{tbl:newtheorem} に示す
ものが用意されています.

\begin{table}[btp]
\caption{$\backslash$\texttt{newtheorem}の見出し}
\label{tbl:newtheorem}
\resizebox{8cm}{!}{%
\begin{tabular}{ll}
\hline
\bsl{newtheorem} の宣言 & 出力例 \\
\hline
\bsl{newtheorem\{definition\}\{定義\}} & \bf 【定義1】 \\
\bsl{newtheorem\{definition\}\{Definition\}} & \bf [Definition 1] \\
\hline
\bsl{newtheorem\{theorem\}\{定理\}} & \bf [定理1] \\
\bsl{newtheorem\{theorem\}\{Theorem\}} & \bf [Theorem 1] \\
\hline
\bsl{newtheorem\{proof\}\{証明\}} & 《証明》$^{*}$ \\
\bsl{newtheorem\{proof\}\{Proof\}} & \bf \hbox{$\LL$}Proof\kern.25ex \hbox{$\GG$}$^{*}$ \\
\hline
\bsl{newtheorem\{lemma\}\{補題\}} & \bf [補題1] \\
\bsl{newtheorem\{lemma\}\{Lemma\}} & \bf [Lemma 1] \\
\hline
\bsl{newtheorem\{corollary\}\{系\}} & \bf (系1) \\
\bsl{newtheorem\{corollary\}\{Corollary\}} & \bf (Corollary 1) \\
\hline
\bsl{newtheorem\{example\}\{例\}} & 〔例1〕 \\
\bsl{newtheorem\{example\}\{Example\}} & \bf [Example 1] \\
\hline
\bsl{newtheorem\{proposition\}\{命題\}} & 〈命題1〉 \\
\bsl{newtheorem\{proposition\}\{Proposition\}} & 
\bf \hbox{$\langle$}Proposition 1\hbox{$\rangle$} \\
\hline
\bsl{newtheorem\{assumption\}\{仮定\}} & \bf [仮定1] \\
\bsl{newtheorem\{assumption\}\{Assumption\}} & \bf [Assumption 1] \\
\hline
\multicolumn{2}{l}{$^{*}$ 番号が付きません.}
\end{tabular}}
\end{table}

\section{参考文献}\label{sec:bib}

\subsection{\BibTeX{}を使わない場合}\label{not-bib}

本誌の \verb/\bibitem/ の記述は以下のとおりです.
\begin{verbfbox}
\bibitem[Deb 02]{ec:02} K. Deb, S. Pratap, ...
\end{verbfbox}

掲載順は,和文・英文の文献を含めて,アルファベット順としてください.

引用は[Goldberg 89]のように著者名と年の間に空白を入れてください.
文献を複数引用する場合は,[Goldberg 89, Deb 02]のようにまとめてください.

\subsection{\BibTeX{}を使う場合}\label{on-bib}

\BibTeX{}用のスタイルファイルを使う場合は
一緒に配布される専用のスタイルファイル\ltt{jpnsec.bst}を使ってください.

参考文献の所定の箇所に次のように指定してください.
\begin{verbfbox}
\bibliography{btxsample} %% .bibファイル名
\bibliographystyle{jpnsec} %% jpnsec.bst スタイルの指定
\end{verbfbox}


データベース\ltt{.bib}ファイルの例を\ref{fig:bibsample}に示しま
す.


\begin{figure}[btp]
\begin{verbFbox}

@Book{goldberg:89,
  author = "D. E. Goldberg",
  title = "Genetic Algorithms in Search,
  Optimization and Machine Learning",
  publisher = "Addison-Wesley Publishing
  Company Inc",
  year = 1989
}

@Article{ec:02,
  author = "K. Deb, S. Pratap, S. Agawal,
  and T. Meyarivan",
  title = "A Fast Elitist Multi-objective
  Genetic Algorithm: NSGA-II",
  journal = "IEEE Trans. on Evolutionary
  Computation",
  year = 2002,
  volume = 6,
  number = 2,
  pages = "181--197"
}

@InProceedings{ppsn:04,
  author = "N. Hansen S. Kern",
  title = "Evaluating the CMA Evolution
  Strategy on Multimodal Test Function",
  booktitle = "Proc. of the 8th Int. Conf.
  on Parallel Problem Solving form Nature
  (PPSN VIII)",
  year = 2004,
  pages = "282--291"
}

@Book{iba:05,
  author = "伊庭 斉志",
  title = "進化論的計算手法",
  publisher = "オーム社",
  year = 2005
}

@Article{jsai:09,
  author = "小林 重信",
  title = "実数値GAのフロンティア",
  yomi = "Kobayashi",
  journal = "人工知能学会論文誌",
  year = 2009,
  volume = 24,
  number = 1,
  pages = "147--162",
}

\end{verbFbox}
\caption{\ltt{.bib}ファイルの例}
\label{fig:bibsample}
\end{figure}


詳しくは\BibTeX{}や,\JBibTeX{}のドキュメント\cite{松井,btxhak}を
参照してください.

\section{採録決定後のファイルの提出について}\label{submit}

投稿時の原稿およびファイルの提出については「原稿執筆案内」を参照してください.
ここでは,採録決定後にファイルを提出するの際の留意点を述べます.

\begin{itemize}
\item
原稿の \TeX{} ファイルは,メインのファイルにインクルードまたは
インプットするのではなく,必ず一本のファイルにまとめてください.
\item
著者独自のマクロなど,コンパイルに必要なソースは必ず添付してください.
\item
一般サイトにない特殊なパッケージを使ったときは,必ずスタイルファイルを添付してください.
ただし,最終組版の段階でそれらパッケージが使えない場合もあることをご承知おきください.
\item
図の ps および eps ファイル,\BibTeX{}の生成する bbl ファイル
も必ず添付してください.
\item
原稿全体をフォーマットしたのちPDFファイルに変換したものを添付して
ください.
\end{itemize}

\bibliography{btxsample}
\bibliographystyle{jpnsec}

\appendix


\section{\texttt{profile-2e.sty}について}

\ltt{profile-2e.sty} は最終組版作成のため事務局側が用いるものですが,
参考のために,説明しておきます.
これは,著者紹介に写真を取り込むためのスタイルファイルで,
\bsl{usepackage} によって \ltt{graphicx} パッケージと共に用いると,
\bsl{profile} に写真を取り込む引数が追加されます.
次のように記述すると, 
\begin{verbfbox}
\begin{biography}
\profile{m}{進化 太郎}{著者の略歴}{portrait}
\end{biography}
\end{verbfbox}
\ltt{portrait.eps}(拡張子は小文字)という
PSファイルが取り込まれます.\ltt{portrait}が写真のファイ
ル名に合わせて変更されます.写真の比率は縦:横が $6:5$ です.


\end{document}
